\documentclass[12pt,a4paperpaper,authoryear]{elsarticle} %review=doublespace preprint=single 5p=2 column
%%% Begin My package additions %%%%%%%%%%%%%%%%%%%
\usepackage[hyphens]{url}
\usepackage{lineno} % add
\usepackage{setspace}
\onehalfspacing

\usepackage{mathpazo}

\providecommand{\tightlist}{%
  \setlength{\itemsep}{0pt}\setlength{\parskip}{0pt}}

\biboptions{sort&compress} % For natbib
\usepackage{graphicx}
\usepackage{booktabs} % book-quality tables
%% Redefines the elsarticle footer
%\makeatletter
%\def\ps@pprintTitle{%
% \let\@oddhead\@empty
% \let\@evenhead\@empty
% \def\@oddfoot{\it \hfill\today}%
% \let\@evenfoot\@oddfoot}
%\makeatother

% A modified page layout
\textwidth 6.75in
\oddsidemargin -0.15in
\evensidemargin -0.15in
\textheight 9in
\topmargin -0.5in
%%%%%%%%%%%%%%%% end my additions to header

\usepackage[T1]{fontenc}
\usepackage{amssymb,amsmath}
\usepackage{ifxetex,ifluatex}
\usepackage{fixltx2e} % provides \textsubscript
% use upquote if available, for straight quotes in verbatim environments
\IfFileExists{upquote.sty}{\usepackage{upquote}}{}
\ifnum 0\ifxetex 1\fi\ifluatex 1\fi=0 % if pdftex
  \usepackage[utf8]{inputenc}
\else % if luatex or xelatex
  \usepackage{fontspec}
  \ifxetex
    \usepackage{xltxtra,xunicode}
  \fi
  \defaultfontfeatures{Mapping=tex-text,Scale=MatchLowercase}
  \newcommand{\euro}{€}
\fi
% use microtype if available
\IfFileExists{microtype.sty}{\usepackage{microtype}}{}
\usepackage[margin=1in]{geometry}
\usepackage{natbib}
\bibliographystyle{model5-names}
\usepackage{longtable}
\ifxetex
  \usepackage[setpagesize=false, % page size defined by xetex
              unicode=false, % unicode breaks when used with xetex
              xetex]{hyperref}
\else
  \usepackage[unicode=true, colorlinks]{hyperref}
\fi
\setlength{\parindent}{0pt}
\setlength{\parskip}{6pt plus 2pt minus 1pt}
\setlength{\emergencystretch}{3em}  % prevent overfull lines
\setcounter{secnumdepth}{5}
% Pandoc toggle for numbering sections (defaults to be off)
% Pandoc header


\usepackage[nomarkers]{endfloat}

%% My customizations %%%%%%%%
%% Loading Packages
\usepackage[inline]{enumitem}
\usepackage{float}
\usepackage{xfrac}
\usepackage{tabularx}
\usepackage{tabularx}
\usepackage[dvipsnames]{xcolor}
\usepackage{pgf, tikz}
\usepackage{amsthm}
\newtheorem{mydef}{Definition}

%% Custom macros
\newcommand{\bs}[1]{\ensuremath{\boldsymbol{#1}}}
\newcommand{\diag}[1]{\mathrm{diag}\left(#1\right)}
\newcommand{\seq}[3][1]{\ensuremath{#2_{#1},\ldots,\,#2_{#3}}}
\newcommand{\note}[1]{\marginpar{\scriptsize\tt{\color{RoyalBlue}#1}}}
\newcommand{\edit}[1]{{\color{OrangeRed} #1}}

%% Custom Lengths
\setlength{\parindent}{0pt}
\setlength{\parskip}{9pt}

%% Define Colors
\newcommand\myshade{85}
\colorlet{mylinkcolor}{violet}
\colorlet{mycitecolor}{YellowOrange}
\colorlet{myurlcolor}{Aquamarine}

%% Hyperlink Setup
\AtBeginDocument{%
  \hypersetup{breaklinks=true,
              bookmarks=true,
              pdfauthor={},
              pdftitle={simrel-m -- A versatile tool for data simulation for multi-response linear model data based on the concept of relevant subspace of predictor space},
              colorlinks=true,
              urlcolor=myurlcolor!\myshade!black,
              linkcolor=mylinkcolor!\myshade!black,
              citecolor=mycitecolor!\myshade!black,
              pdfborder={0 0 0}}
}
% \urlstyle{same}  % don't use monospace font for urls
% 
%% End my customizations %%%%%%%%%

\begin{document}
\begin{frontmatter}

  \title{simrel-m -- A versatile tool for data simulation for multi-response
linear model data based on the concept of relevant subspace of predictor
space}
    \author[Norwegian University of Life Sciences]{Raju Rimal}
  
  
    \author[Norwegian University of Life Sciences]{Trygve Almøy}
  
  
    \author[Norwegian University of Life Sciences]{Solve Sæbø\corref{c1}}
  
   \cortext[c1]{Dep. of Chemistry and Food Science, NMBU, Ås
(\href{http://nmbu.no}{nmbu.no})}
    
  \begin{abstract}
  \small
  Data science is generating enormous amounts of data, and new and
  advanced analytical methods are constantly being developed to cope with
  the challenge of extracting information from such ``big-data''.
  Researchers often use simulated data to assess and document the
  properties of these new methods, and in this paper we present
  \texttt{simrel-m}, which is a versatile and transparent tool for
  simulating linear model data with extensive range of adjustable
  properties. The method is based on the concept of relevant components
  \citet{helland1994comparison}, which is equivalent to the envelope model
  \citet{cook2013envelopes}. It is a multi-response extension of
  \texttt{simrel} \citet{saebo2015simrel}, and as \texttt{simrel} the new
  approach is essentially based on random rotations of latent relevant
  components to obtain a predictor matrix \(\mathbf{X}\), but in addition
  we introduce random rotations of latent components spanning a response
  space in order to obtain a multivariate response matrix \(\mathbf{Y}\).
  The properties of the linear relation between \(\mathbf{X}\) and
  \(\mathbf{Y}\) are defined by a small set of input parameters which
  allow versatile and adjustable simulations. Sub-space rotations also
  allow for generating data suitable for testing variable selection
  methods in multi-response settings. The method is implemented as an
  R-package which serves as an extension of the existing \texttt{simrel}
  packages \citet{saebo2015simrel}.
  \end{abstract}
   \begin{keyword} \footnotesize \texttt{simrel-2.0}, \texttt{simrel} package in r, data
simulation, linear model \sep \end{keyword}
 \end{frontmatter}

\section{Acknoledgement}\label{acknoledgement}

\section{Introduction}\label{introduction}

\emph{General aspects}

Technological advancement has opened a door for complex and
sophisticated scientific experiments that was not possible before. Due
to this change, enormous amounts of raw data are generated which
contains massive information but difficult to excavate. Finding
information and performing scientific research on these raw data is now
becoming another problem. In order to tackle this situation new methods
are being developed. However, before implementing any methods, it is
essential to test its performance. Often, researchers use simulated data
for the purpose which itself is a time-consuming process. The main focus
of this paper is to present a simulation method, along with an r-package
called \texttt{simrel-m}, that is versatile in nature and yet simple to
use.

The method is based on principal of relevant space for prediction which
assumes that there exists a subspace in the complete space of responses
that is spanned by a subset of eigenvectors of predictor variables. The
method and the r-package based on this principle not only has ability to
simulate wide range of multi-response linear model data but also let
researcher to specify which components of predictors \((\mathbf{X})\)
are relevant for a component of responses \(\mathbf{Y}\). This enables
the possibility to construct data for evaluating methods developed for
variable selection.

A vast literature on simulation is present but most of them are
developed to address the specific problems their study was dealing with.
\citet{saebo2015simrel} has presented a generic tool that is capable of
simulating linear model data as an r-package \texttt{simrel}. This paper
extends the methods to simulate multivariate response.

\emph{Application of \texttt{simrel-m}}

The simple interface of \texttt{simrel-m} for sophisticated simulation
opens for numerous applications in various disciplines among which some
of them are discussed below.

\begin{description}
\tightlist
\item[\emph{Educational Purpose:}]
Explaining multivariate statistics and edify people is a difficult and
strategic task. Instructors spend lots of time just to find suitable
datasets for explaining some issue. For instance --
\item[\emph{Model and methods testing:}]
Imagine a situation where a researcher is collecting enormous amount of
data which takes both time and money. Before going into extensive
sampling, a pilot project is started and samples were collected using
various techniques. Each of them are modelled with different estimation
methods. The researcher would like to compare the sampling methods or
estimation procedures that will the suitable for the final project. A
simulated dataset based on the pilot project may help to identify the
most appropriate sampling methods or estimation procedures.
\item[\emph{Understanding and developing multivariate statistics:}]
New estimation methods are being developed to address mordern and
complicated situations. A new method or technique could be difficult to
understand. For example, the envelope model \citet{cook2015foundations},
a recent estimation technique based on maximum likelihood, attempts to
find a response envelope (relevant subspace) that contains all the
information that the corresponding predictors can explain. Here, one can
make use of \texttt{simrel-m} to simulate data with underlying
informative response space. In addition, new methods such as CPLS based
on PLS are steadily being developed. Understanding population latent
structure enables assessment of such models.
\end{description}

\subsection{Model Specification}\label{model-specification}

A multi-response multivariate general linear model in
equation-\eqref{eq:model1} is conidered as a simulation model.

Being an extension of \texttt{simrel} package, a quick summary of the
procedure used in that package helps to underestand the literature in
this paper.

\subsubsection{\texorpdfstring{An overview of
\texttt{simrel}}{An overview of simrel}}\label{an-overview-of-simrel}

\texttt{Simrel} is based on uni-response linear model as in
equation\textasciitilde{}\eqref{eq:simrel-model}.

\begin{equation}
\label{eq:simrel-model}
  \begin{bmatrix}
    y \\ \mathbf{X}
  \end{bmatrix} \sim
  \mathcal{N}\left(
    \begin{bmatrix}
      \mu_y \\ \boldsymbol{\mu_X}
    \end{bmatrix},
    \begin{bmatrix}
      \sigma_y^2               & \boldsymbol{\sigma_{Xy}}^t \\
      \boldsymbol{\sigma_{Xy}} & \boldsymbol{\Sigma_{XX}}
    \end{bmatrix}
  \right)
\end{equation}

\begin{equation}
  \mathbf{Y} = \boldsymbol{\mu}_Y + \boldsymbol{B}^t (\mathbf{X} - \boldsymbol{\mu}_X) + \boldsymbol{\epsilon}
  \label{eq:model1}
\end{equation}

where \(\mathbf{Y}\) is a response matrix with \(m\) response vector
\(y_1, y_2, \ldots y_m\), \(\mathbf{X}\) is multivariate predictor
matrix with \(p\) predictor variables and the random error term
\(\boldsymbol{\epsilon}\) is assumed to follow
\(N(\boldsymbol{0},\; \boldsymbol{\Sigma}_{Y|X})\). Equivalently,

\begin{equation}
  \begin{bmatrix}\mathbf{Y}\\ \mathbf{X}\end{bmatrix} \sim N(\boldsymbol{\mu}, \boldsymbol{\Sigma})
  = N \left(
    \begin{bmatrix}
      \boldsymbol{\mu}_Y \\
      \boldsymbol{\mu}_X
    \end{bmatrix},
    \begin{bmatrix}
      \boldsymbol{\Sigma}_{YY} & \boldsymbol{\Sigma}_{XY}^t \\
      \boldsymbol{\Sigma}_{XY} & \boldsymbol{\Sigma}_{XX}
    \end{bmatrix}
  \right)
  \label{eq:model2}
\end{equation}

Here,

\(\boldsymbol{\Sigma}_{YY}\) : Covariance Matrix of response
\(\mathbf{Y}\) without given \(\mathbf{X}\)\\
\(\boldsymbol{\Sigma}_{XY}\) : Covariance Matrix between \(\mathbf{X}\)
and \(\mathbf{Y}\)\\
\(\boldsymbol{\Sigma}_{XX}\) : Covariance matrix of predictor variables
\(\mathbf{X}\)\\
\(\boldsymbol{\mu}_X\) and \(\boldsymbol{\mu}_Y\) : Mean vectors of
response \(\mathbf{Y}\) and predictor \(\mathbf{X}\) respective

According to the theory of Multivariate Normal Distribution, we can
express different parameters interms of \(\mathbf{X}\), \(\mathbf{Y}\)
and the covariance structure.

\subsubsection{Model Parameterization}\label{model-parameterization}

\texttt{Simrel-m} uses model parameterization which is based on the
concept of relevant components \citet{helland1994comparison} where it is
assumed that a subspace of response \(\mathbf{Y}\) is spanned by a
subset of eigenvectors corresponding to predictor space. A response
space can be thought to have two mutually orthogonal space -- relevant
and irrelevant. Here the relevant space of response matrix is termed as
response components, and we assume that each response component is
spanned by an exclusive subset of predictor variables. In this way we
can construct a set of predictor variables which has non-zero regression
coefficients. This also enables user to have uninformative predictors
which can be detected during variable selection procedure. In addition,
user can control signal-to-noise ratio for each response components with
a vector of population coefficient of determination
\(\rho_1, \ldots, \rho_q\). Further, the collinearity between predictor
variables can also be controlled by a factor \(\gamma\) which guides the
decay pattern of eigenvalue of \(\mathbf{X}\) matrix.
\citet{helland1994comparison} showed that if the direction of large
variablity (i.e., component corresponding to large eigenvalues) are also
relevant relevant predictor space, prediction is relatively easy. In
contrast, if the relevant predictors are on the direction of low
varibility, prediction becomes difficult.

\emph{Parameter Definition:}

Before continuing any further, it is necessary to define the parameters
used here,

\section{Stat Model}\label{stat-model}

\bibliography{packages.bib,ref-db.bib}

\end{document}
