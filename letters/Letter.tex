\documentclass[11pt,a4paper,]{letter}
\usepackage[]{mathpazo}
\usepackage{setspace}
\setstretch{1.15}
\usepackage{amssymb,amsmath}
\usepackage{ifxetex,ifluatex}
\usepackage{fixltx2e} % provides \textsubscript
\ifnum 0\ifxetex 1\fi\ifluatex 1\fi=0 % if pdftex
  \usepackage[T1]{fontenc}
  \usepackage[utf8]{inputenc}
\else % if luatex or xelatex
  \ifxetex
    \usepackage{mathspec}
  \else
    \usepackage{fontspec}
  \fi
  \defaultfontfeatures{Ligatures=TeX,Scale=MatchLowercase}
\fi
% use upquote if available, for straight quotes in verbatim environments
\IfFileExists{upquote.sty}{\usepackage{upquote}}{}
% use microtype if available
\IfFileExists{microtype.sty}{%
\usepackage{microtype}
\UseMicrotypeSet[protrusion]{basicmath} % disable protrusion for tt fonts
}{}
\usepackage[margin=1in]{geometry}
\usepackage[unicode=true]{hyperref}
\PassOptionsToPackage{usenames,dvipsnames}{color} % color is loaded by hyperref
\hypersetup{
            colorlinks=true,
            linkcolor=Maroon,
            citecolor=Blue,
            urlcolor=Blue,
            breaklinks=true}
\urlstyle{same}  % don't use monospace font for urls
% Make links footnotes instead of hotlinks:
\renewcommand{\href}[2]{#2\footnote{\url{#1}}}
\IfFileExists{parskip.sty}{%
\usepackage{parskip}
}{% else
\setlength{\parindent}{0pt}
\setlength{\parskip}{6pt plus 2pt minus 1pt}
}
\setlength{\emergencystretch}{3em}  % prevent overfull lines
\providecommand{\tightlist}{%
  \setlength{\itemsep}{0pt}\setlength{\parskip}{0pt}}
\setcounter{secnumdepth}{0}
% Redefines (sub)paragraphs to behave more like sections
\ifx\paragraph\undefined\else
\let\oldparagraph\paragraph
\renewcommand{\paragraph}[1]{\oldparagraph{#1}\mbox{}}
\fi
\ifx\subparagraph\undefined\else
\let\oldsubparagraph\subparagraph
\renewcommand{\subparagraph}[1]{\oldsubparagraph{#1}\mbox{}}
\fi

% set default figure placement to htbp
\makeatletter
\def\fps@figure{htbp}
\makeatother

% \signature{Raju Rimal,\\PhD Fellow\\NMBU\\Faculty of Chemistry and Biotechnology\\\AA s, Norway\\\vspace{1.5cm}Solve S\ae b\o,\\Professor\\NMBU\\Faculty of Chemistry and Biotechnology\\\AA s, Norway\\\vspace{1.5cm}Trygve Alm\o y,\\Associate Professor\\NMBU\\Faculty of Chemistry and Biotechnology\\\AA s, Norway}


% % % 
\date{July 25, 2017}

\address{Raju Rimal\\Vollveien 7, 1433, \AA s\\Norway}

\usepackage{mdframed} % color is loaded by mdframed
\definecolor{greyborder}{RGB}{221,221,221}
\definecolor{greytext}{RGB}{119,119,119}
\newmdenv[rightline=false,bottomline=false,topline=false,linewidth=3pt,linecolor=greyborder,skipabove=\parskip]{blockquote}
\renewenvironment{quote}{\begin{blockquote}\list{}{\rightmargin=0em\leftmargin=0em}%
\item\relax\color{greytext}\ignorespaces}{\unskip\unskip\endlist\end{blockquote}}


\begin{document}


\begin{letter}{Romà Tauler,\\Editor in Chief\\Chemometrics and Intelligent Laboratory Systems}
\opening{Dear Mr.~Tauler,}

We are pleased to submit an original research article entitled
``\textbf{A tool for simulating multi-response linear model data}''.
This manuscript is an extension to r-package ``\emph{simrel}''
\footnote{S. Sæbø, T. Almøy and I. S. Helland. ``simrel - A versatile
  for linear model data simulation based on the concept of avant
  subspace and relevant predictors''. In: \emph{Chemometrics andlligent
  Laboratory Systems} (2015).} previously published in Chemometrics and
Intelligent Laboratory System. The extension is to incorporate
simulation of multiple response variable.

Many methods have been developed for handling multivariate data, and
usually one is using simulated data to access various properties of the
methods. The tool presented in this manuscript will accelerate the
process by simulating multi-response multivariate data. This enables
researchers not only to study method-data interaction but also
facilitate them to make extensive comparison with other similar methods.

We believe that this manuscript is appropriate for publication on
\emph{software description} section on Chemometrics and Intelligent
Laboratory System. Many research articles in chemometrics and related
fields have multi-response variables with few underlying latent
structure. Since the tool described in this manuscript is based on the
underlying latent structure, researchers can simulate data with control
over few parameters. We believe that this tool can help researchers in
various parts of their research process.

This manuscript has not been published and is not under consideration
for publication elsewhere. We have no conflict of interest to disclose.

Thank you for your consideration.

\longindentation=0pt
\closing{Sincerely,}
 \parbox[b]{0.3\textwidth}{Raju Rimal,\\PhD Fellow\\NMBU\\Faculty of Chemistry and Biotechnology\\\AA s, Norway}\hfill \parbox[b]{0.3\textwidth}{Solve S\ae b\o,\\Professor\\NMBU\\Faculty of Chemistry and Biotechnology\\\AA s, Norway}\hfill \parbox[b]{0.3\textwidth}{Trygve Alm\o y,\\Associate Professor\\NMBU\\Faculty of Chemistry and Biotechnology\\\AA s, Norway}

\end{letter}

\end{document}
