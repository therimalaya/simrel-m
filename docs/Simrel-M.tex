\documentclass[12pt,A4paper,authoryear]{elsarticle} %review=doublespace preprint=single 5p=2 column
%%% Begin My package additions %%%%%%%%%%%%%%%%%%%
%% Other Customizations
\usepackage[hyphens]{url}
\usepackage{lineno} % add
\usepackage{caption}


\usepackage{setspace}
\setstretch{1.25}

\usepackage{mathpazo}


\providecommand{\tightlist}{%
  \setlength{\itemsep}{0pt}\setlength{\parskip}{0pt}}

\biboptions{sort&compress} % For natbib
\usepackage{booktabs} % book-quality tables

% Redefines the elsarticle footer
\makeatletter
\def\ps@pprintTitle{%
  \let\@oddhead\@empty
  \let\@evenhead\@empty
  \def\@oddfoot{\it \hfill\today}%
  \let\@evenfoot\@oddfoot}
\makeatother

% A modified page layout
\textwidth 6.75in
\oddsidemargin -0.15in
\evensidemargin -0.15in
\textheight 9in
\topmargin -0.5in
%%%%%%%%%%%%%%%% end my additions to header

\usepackage[T1]{fontenc}
\usepackage{amssymb,amsmath}

\usepackage{ifxetex,ifluatex}
\usepackage{fixltx2e} % provides \textsubscript
% use upquote if available, for straight quotes in verbatim environments
\IfFileExists{upquote.sty}{\usepackage{upquote}}{}
\ifnum 0\ifxetex 1\fi\ifluatex 1\fi=0 % if pdftex
\usepackage[utf8]{inputenc}

\usepackage[T1]{fontenc}
\usepackage{amssymb,amsmath}
\usepackage{ifxetex,ifluatex}
\usepackage{fixltx2e} % provides \textsubscript
% use upquote if available, for straight quotes in verbatim environments
\IfFileExists{upquote.sty}{\usepackage{upquote}}{}
\ifnum 0\ifxetex 1\fi\ifluatex 1\fi=0 % if pdftex
\usepackage[utf8]{inputenc}
\else % if luatex or xelatex
\usepackage{fontspec}
\ifxetex
\usepackage{xltxtra,xunicode}
\fi
\defaultfontfeatures{Mapping=tex-text,Scale=MatchLowercase}
\newcommand{\euro}{€}
\setmonofont{Source Code Pro}
\fi
% use microtype if available
\IfFileExists{microtype.sty}{\usepackage{microtype}}{}
\usepackage[margin=1in]{geometry}
\usepackage{natbib}
\bibliographystyle{plainnat}
\usepackage{longtable}
\ifxetex
\usepackage[setpagesize=false, % page size defined by xetex
unicode=false, % unicode breaks when used with xetex
xetex]{hyperref}
\else
\usepackage[unicode=true]{hyperref}
\fi

\setlength{\parindent}{0pt}
\setlength{\parskip}{6pt plus 2pt minus 1pt}
\setlength{\emergencystretch}{3em}  % prevent overfull lines
\setcounter{secnumdepth}{0}
% Pandoc toggle for numbering sections (defaults to be off)
\setcounter{secnumdepth}{0}
% Pandoc header


% \usepackage[nomarkers]{endfloat}

%% My customizations %%%%%%%%
%% Loading Packages
\usepackage[inline]{enumitem}
\usepackage{float}
\usepackage{tabularx}
\usepackage[dvipsnames]{xcolor}
\usepackage{pgf, tikz}
\usepackage{amsthm}

%% Custom macros
\newtheorem{mydef}{Definition}
\newcommand{\bs}[1]{\ensuremath{\boldsymbol{#1}}}
\newcommand{\diag}[1]{\mathrm{diag}\left(#1\right)}
\newcommand{\seq}[3][1]{\ensuremath{#2_{#1},\ldots,\,#2_{#3}}}
\newcommand{\note}[1]{\marginpar{\scriptsize\tt{\color{RoyalBlue}#1}}}
\newcommand{\edit}[1]{{\color{OrangeRed} #1}}

%% Custom Lengths
\setlength{\parindent}{0pt}
\setlength{\parskip}{9pt}

%% Define Colors
\newcommand\myshade{85}
\colorlet{mylinkcolor}{violet}
\colorlet{mycitecolor}{YellowOrange}
\colorlet{myurlcolor}{Aquamarine}

%% Hyperlink Setup
\AtBeginDocument{%
  \hypersetup{breaklinks=true,
    bookmarks=true,
    pdfauthor={},
    pdftitle={simrel-m: A versatile tool for simulating multi-response linear model data},
    colorlinks=true,
    urlcolor=myurlcolor!\myshade!black,
    linkcolor=mylinkcolor!\myshade!black,
    citecolor=mycitecolor!\myshade!black,
    pdfborder={0 0 0}}
}

\urlstyle{same}  % don't use monospace font for urls

%% End my customizations %%%%%%%%%

\begin{document}
\begin{frontmatter}

  \title{\texttt{simrel-m}: A versatile tool for simulating multi-response linear
model data}
    \author[]{Raju Rimal}
  
  
    \author[]{Trygve Almøy}
  
  
    \author[Norwegian University of Life Sciences]{Solve Sæbø\corref{c1}}
  
   \cortext[c1]{Dep. of Chemistry and Food Science, NMBU, Ås
(\href{http://nmbu.no}{nmbu.no})}
    
  \begin{abstract}
    \small
    Data science is generating enormous amounts of data, and new and
    advanced analytical methods are constantly being developed to cope with
    the challenge of extracting information from such ``big-data''.
    Researchers often use simulated data to assess and document the
    properties of these new methods, and in this paper we present
    \texttt{simrel-m}, which is a versatile and transparent tool for
    simulating linear model data with extensive range of adjustable
    properties. The method is based on the concept of relevant components
    \citet{helland1994comparison}, which is equivalent to the envelope model
    \citet{cook2013envelopes}. It is a multi-response extension of
    \texttt{simrel} \citet{saebo2015simrel}, and as \texttt{simrel} the new
    approach is essentially based on random rotations of latent relevant
    components to obtain a predictor matrix \(\mathbf{X}\), but in addition
    we introduce random rotations of latent components spanning a response
    space in order to obtain a multivariate response matrix \(\mathbf{Y}\).
    The properties of the linear relation between \(\mathbf{X}\) and
    \(\mathbf{Y}\) are defined by a small set of input parameters which
    allow versatile and adjustable simulations. Sub-space rotations also
    allow for generating data suitable for testing variable selection
    methods in multi-response settings. The method is implemented as an
    R-package which serves as an extension of the existing \texttt{simrel}
    packages \citet{saebo2015simrel}.
  \end{abstract}
   \begin{keyword} \footnotesize \texttt{simrel-2.0}, \texttt{simrel} package in r, data
simulation, linear model, \texttt{simrel-m} \sep \end{keyword}
\end{frontmatter}

\subsection{Introduction}\label{introduction}

Technological advancement has opened a door for complex and
sophisticated scientific experiments that was not possible before. Due
to this change, enormous amounts of raw data are generated which
contains massive information but difficult to excavate. Finding
information and performing scientific research on these raw data has now
become another problem. In order to tackle this situation new methods
are being developed. However, before implementing any method, it is
essential to test its performance. Often, researchers use simulated data
for the purpose which itself is a time-consuming process. The main focus
of this paper is to present a simulation method, along with an r-package
called \texttt{simrel-m}, that is versatile in nature and yet simple to
use.

The simulation method we are discussing here is based on principal of
relevant space for prediction \citep{helland1994comparison} which
assumes that there exists a subspace in the complete space of response
variables that is spanned by a subset of eigenvectors of predictor
variables. The r-package based on this method lets user to specify
various population properties such as which components of predictors
\((\mathbf{X})\) are relevant for a component of responses
\(\mathbf{Y}\) and how the eigenvalues of \(\mathbf{X}\) decreases. This
enables the possibility to construct data for evaluating estimation
methods and methods developed for variable selection.

Among several literatures in simulation
({\color{red}\texttt{which\ literatures}}), \citet{ripley2009stochastic}
has exhaustively discussed the topic. In addition, many literatures
({\color{red}\texttt{which\ literatures}}) are available on studies
which has implemented simulated data in order to investigate new
estimation methods and prediction strategy
\citep[see:][]{cook2015simultaneous, cook2013envelopes, helland2012near}.
However, most of the simulations in these studies is developed to
address their specific problem. A systematic tool for simulating linear
model data with single response, which could serve as a general tool for
all such comparisons, was presented in \citet{saebo2015simrel} and as
r-package \texttt{simrel}. This paper extends \texttt{simrel} in order
to simulate linear model data with multivariate response with an
r-package \texttt{simrel-m}.

The r-package \texttt{simrel-m} uses model parameterization which is
based on the concept of relevant components
\citet{helland1994comparison} where it is assumed that a subspace of
response \(\mathbf{Y}\) is spanned by a subset of eigenvectors
corresponding to predictor space. A response space can be thought to
have two mutually orthogonal space -- relevant and irrelevant. Here the
space of response matrix for which the predictors are relevant is termed
as response components, and we assume that each response component is
spanned by an exclusive subset of predictor variables. In this way we
can construct a set of predictor variables which has non-zero regression
coefficients. This also enables user to have uninformative predictors
which can be detected during variable selection procedure. In addition,
user can control signal-to-noise ratio for each response components with
a vector of population coefficient of determination
\(\boldsymbol{\rho}^2\). Further, the collinearity between predictor
variables can also be adjusted by a factor \(\gamma\) which controls
decay factor of eigenvalue of \(\mathbf{X}\) matrix.
\citet{helland1994comparison} showed that if the direction of large
variability (i.e., component corresponding to large eigenvalues) are
also relevant relevant predictor space, prediction is relatively easy.
In contrast, if the relevant predictors are on the direction of low
variability, prediction becomes difficult.
Table\textasciitilde{}\ref{tab:parameters} shows all the parameters that
a user can specify in \texttt{simrel-m}.

\begin{longtable}[]{@{}cl@{}}
\caption{\label{tab:parameters} Parameters for simulation used in this
study}\tabularnewline
\toprule
\begin{minipage}[b]{0.19\columnwidth}\centering\strut
Parameters\strut
\end{minipage} & \begin{minipage}[b]{0.75\columnwidth}\raggedright\strut
Description\strut
\end{minipage}\tabularnewline
\midrule
\endfirsthead
\toprule
\begin{minipage}[b]{0.19\columnwidth}\centering\strut
Parameters\strut
\end{minipage} & \begin{minipage}[b]{0.75\columnwidth}\raggedright\strut
Description\strut
\end{minipage}\tabularnewline
\midrule
\endhead
\begin{minipage}[t]{0.19\columnwidth}\centering\strut
\(n\)\strut
\end{minipage} & \begin{minipage}[t]{0.75\columnwidth}\raggedright\strut
number of observations\strut
\end{minipage}\tabularnewline
\begin{minipage}[t]{0.19\columnwidth}\centering\strut
\(p\)\strut
\end{minipage} & \begin{minipage}[t]{0.75\columnwidth}\raggedright\strut
number of predictor variables\strut
\end{minipage}\tabularnewline
\begin{minipage}[t]{0.19\columnwidth}\centering\strut
\(q\)\strut
\end{minipage} & \begin{minipage}[t]{0.75\columnwidth}\raggedright\strut
numbers of relevant predictors for each latent component of response
variables\strut
\end{minipage}\tabularnewline
\begin{minipage}[t]{0.19\columnwidth}\centering\strut
\(l\)\strut
\end{minipage} & \begin{minipage}[t]{0.75\columnwidth}\raggedright\strut
number of informative latent component of response variables (response
components)\strut
\end{minipage}\tabularnewline
\begin{minipage}[t]{0.19\columnwidth}\centering\strut
\(m\)\strut
\end{minipage} & \begin{minipage}[t]{0.75\columnwidth}\raggedright\strut
number of response variables\strut
\end{minipage}\tabularnewline
\begin{minipage}[t]{0.19\columnwidth}\centering\strut
\(\gamma\)\strut
\end{minipage} & \begin{minipage}[t]{0.75\columnwidth}\raggedright\strut
degree of collinearity (factor that control the decrease of eigenvalue
of \(\mathbf{X}\))\strut
\end{minipage}\tabularnewline
\begin{minipage}[t]{0.19\columnwidth}\centering\strut
\(\mathcal{P}\)\strut
\end{minipage} & \begin{minipage}[t]{0.75\columnwidth}\raggedright\strut
position index of relevant predictor components for each response
components\strut
\end{minipage}\tabularnewline
\begin{minipage}[t]{0.19\columnwidth}\centering\strut
\(\mathcal{S}\)\strut
\end{minipage} & \begin{minipage}[t]{0.75\columnwidth}\raggedright\strut
position index of response components to combine relevant and irrelevant
response variable\strut
\end{minipage}\tabularnewline
\begin{minipage}[t]{0.19\columnwidth}\centering\strut
\(\boldsymbol{\rho}^2\)\strut
\end{minipage} & \begin{minipage}[t]{0.75\columnwidth}\raggedright\strut
population coefficient determination for each response components\strut
\end{minipage}\tabularnewline
\bottomrule
\end{longtable}

Based on random regression model in
equation\textasciitilde{}\eqref{eq:model1}, we discuss some of these
parameters in details.

Out of \(p\) predictor variables only \(q\) of them are relevant and has
non-zero regression coefficients. If
\(\boldsymbol{e}_i, \; i = 1, 2, \ldots, p\) be the eigenvectors
corresponding to \(\mathbf{X}\), then, for some vector of
\(\boldsymbol{\eta}_j, j = 1, \ldots, m\) for \(m\) response variables,

\begin{align}
\mathbf{B} = \left(\beta_{ij}\right) &= 
  \begin{bmatrix} 
    \boldsymbol{\eta}_{11} & \ldots & \boldsymbol{\eta}_{1p} \\
    \vdots & \ddots & \vdots \\
    \boldsymbol{\eta}_{m1} & \ldots & \boldsymbol{\eta}_{mp}
  \end{bmatrix} 
  \begin{bmatrix} 
    \mathbf{e}_{11} & \vdots & \mathbf{e}_{1p} \\
    \vdots & \ddots & \vdots \\
    \mathbf{e}_{p1} & \vdots & \mathbf{e}_{pp}
  \end{bmatrix} \\ &= 
  \begin{bmatrix}\boldsymbol{\eta}_{1} \\ \vdots \\ \boldsymbol{\eta}_{m}\end{bmatrix}_{m \times p}
  \begin{bmatrix}\mathbf{e}_{1} & \ldots & \mathbf{e}_{p}\end{bmatrix}_{p \times p} 
\label{eq:beta-eigenvector-relation}
\end{align}

The number of terms in
equation\textasciitilde{}\eqref{eq:beta-eigenvector-relation} may be
reduced by two mechanisms:

\begin{enumerate}
\def\labelenumi{\alph{enumi})}
\tightlist
\item
  Some elements in \(\boldsymbol{\eta}_j, j = 1, \ldots m\) are zero
\item
  There are coinciding eigenvalues of \(\boldsymbol{\Sigma}_{xx}\) such
  that it is enough to have one eigenvector in
  equation\textasciitilde{}\eqref{eq:beta-eigenvector-relation}.
\end{enumerate}

Let there are \(k\) number predictor components that are relevant for
any of the response. The \(\mathcal{P}\) contains the indices of these
position for each response components. Here the order of components of
predictor is defined by a decreasing set of eigenvalues such that
\(\lambda_1 \ge \ldots \ge \lambda_p > 0\). In \texttt{simrel-m}
package, these set of position is referred as \texttt{relpos}. For
example, if \(\mathcal{P} = {{1, 2}, {3, 5}, {4}}\) then we can say that
there are 3 informative space in response such that components 1 and 2
of predictor variable is relevant for first response component;
component 3 and 5 of predictor are relevant for second response
component and fourth predictor component is relevant for third response
component. In addition, \(\lambda_1 \ge \ldots \ge \lambda_5\) are
relevant for some response.

In \texttt{simrel-m}, these 3 response components are combined with
non-informative vectors for desired number of response variables. This
is referred as \(\mathcal{S}\) and \texttt{ypos} in \texttt{simrel-m}
package. If \(\mathcal{S} = {{1, 4}, {2}, {3, 5}}\) then we can say that
there are 5 response variables which has 3 dimensional informative
space. Since response components 1, 2 and 3 are informative, from the
indices of \(\mathcal{S}\), first response component is combined with
non-informative fourth component and third informative component is
combined with fifth non-informative component. In this way, we will
obtain a set of 5 response variables for which predictor component 1 and
2 will be relevant for response 1 and 4; predictor component 3 and 5
will be relevant for response 2 and predictor component 4 will be
relevant for response 3 and 5.

For simplification, an assumption is made that all \(p\) eigenvalues of
\(\boldsymbol{\Sigma}_{XX}\) decrease exponentially as
\(e^{-\gamma (i - 1)}\) for \(i = 1, \ldots p\) and some positive
constant \(\gamma\). This way, the \(p\) eigenvalues depends on single
variables \(\gamma\) such that when \(\gamma\) is large, eigenvalues
decreases sharply referring high degree of multi-collinearity in
predictor variables.

Here we have assumed that the relevant components are know, as in
\citet{helland1994comparison}, which is rare in practice. But in
comparative studies of prediction methods, this can help to explain
interesting cases.

\begin{center}\rule{0.5\linewidth}{\linethickness}\end{center}

Let us consider a random regression model in
equation\textasciitilde{}\eqref{eq:model1} as our point of departure.

\begin{equation}
  \mathbf{Y} = \boldsymbol{\mu}_Y + \mathbf{B}^t (\mathbf{X} - \boldsymbol{\mu}_X) + \boldsymbol{\epsilon}
  \label{eq:model1}
\end{equation}

where \(\mathbf{Y}\) is a response matrix with \(m\) response variables
\(y_1, y_2, \ldots y_m\) with mean vector of \(\boldsymbol{\mu}_Y\);
\(\mathbf{X}\) is vector of \(p\) predictor variables and the random
error term \(\boldsymbol{\epsilon}\) is assumed to follow
\(N(\boldsymbol{0},\; \boldsymbol{\Sigma}_{Y|X})\). In addition, we
assume equation\textasciitilde{}\eqref{eq:model1} as a random regression
model where
\(\mathbf{X} \sim N\left(\boldsymbol{\mu}_X, \boldsymbol{\Sigma}_{XX}\right)\)
independent of \(\boldsymbol{\epsilon}\).

\subsubsection{Model Specification}\label{model-specification}

A multi-response multivariate general linear model in
equation-\eqref{eq:model1} is conidered as a simulation model.

Being an extension of \texttt{simrel} package, a quick summary of the
procedure used in that package helps to underestand the literature in
this paper.

\paragraph{\texorpdfstring{An overview of
\texttt{simrel}}{An overview of simrel}}\label{an-overview-of-simrel}

\texttt{Simrel} is based on uni-response linear model as in
equation\textasciitilde{}\eqref{eq:simrel-model}.

\begin{equation}
\label{eq:simrel-model}
  \begin{bmatrix}
    y \\ \mathbf{X}
  \end{bmatrix} \sim
  \mathcal{N}\left(
    \begin{bmatrix}
      \mu_y \\ \boldsymbol{\mu_X}
    \end{bmatrix},
    \begin{bmatrix}
      \sigma_y^2               & \boldsymbol{\sigma_{Xy}}^t \\
      \boldsymbol{\sigma_{Xy}} & \boldsymbol{\Sigma_{XX}}
    \end{bmatrix}
  \right)
\end{equation}

Equivalently,

\begin{equation}
  \begin{bmatrix}\mathbf{Y}\\ \mathbf{X}\end{bmatrix} \sim N(\boldsymbol{\mu}, \boldsymbol{\Sigma})
  = N \left(
    \begin{bmatrix}
      \boldsymbol{\mu}_Y \\
      \boldsymbol{\mu}_X
    \end{bmatrix},
    \begin{bmatrix}
      \boldsymbol{\Sigma}_{YY} & \boldsymbol{\Sigma}_{XY}^t \\
      \boldsymbol{\Sigma}_{XY} & \boldsymbol{\Sigma}_{XX}
    \end{bmatrix}
  \right)
  \label{eq:model2}
\end{equation}

Here,

\(\boldsymbol{\Sigma}_{YY}\) : Covariance Matrix of response
\(\mathbf{Y}\) without given \(\mathbf{X}\)\\
\(\boldsymbol{\Sigma}_{XY}\) : Covariance Matrix between \(\mathbf{X}\)
and \(\mathbf{Y}\)\\
\(\boldsymbol{\Sigma}_{XX}\) : Covariance matrix of predictor variables
\(\mathbf{X}\)\\
\(\boldsymbol{\mu}_X\) and \(\boldsymbol{\mu}_Y\) : Mean vectors of
response \(\mathbf{Y}\) and predictor \(\mathbf{X}\) respective

According to the theory of Multivariate Normal Distribution, we can
express different parameters interms of \(\mathbf{X}\), \(\mathbf{Y}\)
and the covariance structure.

\paragraph{Model Parameterization}\label{model-parameterization}

\emph{Parameter Definition:}

Before continuing any further, it is necessary to define the parameters
used here,

In the following section, some of these parameters are discussed in
detail. The discussion has considered random \(\mathbf{X}\) regression
model with \(m\) response as given in
equation\textasciitilde{}\eqref{eq:model1}.

\emph{Parameters Explanation and Notation Used:}

Let \(m\) responses are spanned completely by \(l\) response components.
These \(l\) response components are combined with \(m-l\) standard
normal vectors by user defined criteria to get \(m\) responses after
successive orthogonal transformation. Out of \(q\), let
\(q_j,\; j = 1, \ldots l\) be the number of predictors that is relevant
for response \(j\). Let \(c_j\) be the number of eigenvectors/
components that completely span \(j^\text{th}\) predictor space
containing \(q_j\) number of predictors. Further, the position of these
components for \(j^\text{th}\) response be in index set
\(\mathcal{P}_j\). Here it is also assumed that the eigenvalues
corresponding to \(\mathbf{X}\) declines successively such that
\(\lambda_i, i = 1, \ldots, p\) such that
\(\lambda_i \ge \lambda_k, i > k\) are the eigen values of
\(\mathbf{X}\). The position index of eigenvalues corresponding to
response \(j\) is in the set \(\mathcal{P}_j\). We assume that the index
are ordered within each sets so that, \(j^\text{th}\) index set contains
\(c_j\) number of components. The eigenvalues corresponding to these
components in \(\mathcal{P}_j\) set is
\(\lambda_{\mathcal{P}_{jk}}, k = 1, \ldots c_j\) such that
\(\lambda_{\mathcal{P}_{jk}} > \lambda_{\mathcal{P}_{jk'}}\) for
\(k > k'\). In \texttt{Simrel-M} package, we refer this position by
\texttt{relpos} argument. In addition, we suppose that the relevant
components are exclusive for each response.

\emph{An Example:}

Suppose we have a situation like,

\begin{longtable}[]{@{}rlcc@{}}
\toprule
Number of response & \((m)\) & = & 5\tabularnewline
Number of response components & \((l)\) & = & 3\tabularnewline
Position of relevant component for response 1 & \((\mathcal{P}_1)\) & =
& \(\left\{ 1, 3 \right\}\)\tabularnewline
Position of relevant component for response 2 & \((\mathcal{P}_2)\) & =
& \(\left\{ 2, 4, 5 \right\}\)\tabularnewline
Position of relevant component for response 3 & \((\mathcal{P}_3)\) & =
& \(\left\{ 6 \right\}\)\tabularnewline
\bottomrule
\end{longtable}

such that, \(\lambda_1 > \lambda_3\) in \(\mathcal{P}_1\) and
\(\lambda_2 > \lambda_4 > \lambda_5\) in set \(\mathcal{P}_2\). Here,
the component (eigenvector) 1 and 3 are relevant for response component
1, component 2, 4 and 5 are relevant for response component 2 and
component 6 is relevant for response component 3.

In \texttt{Simrel-M}, we have assumed that the eigenvalues are
decreasing exponentially by factor \(\gamma\) and the largest eigenvalue
is 1, i.e.~for \(\gamma > 0\), \(\lambda_i = \text{e}^{-\gamma(i-1)}\)
for \(i = 1, 2, \ldots p\).

\section*{References}\label{references}
\addcontentsline{toc}{section}{References}


\renewcommand\refname{References}
\bibliography{packages.bib,ref-db.bib}



\end{document}
